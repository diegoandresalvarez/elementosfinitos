% !TeX spellcheck = en_US
% !TeX document-id = {cf032f19-4427-4a50-b0ba-ddee651fb980}
% !TeX program = txs:///dvi-ps-pdf-chain
% TeX program = txs:///dvi-chain
%DVI -> PS -> PDF
% 10.1016/j.compstruc.2017.04.006 
% OJO VERIFICAR IDIOMA GB

%\documentclass[preprint,12pt,authoryear,letterpaper,times]{elsarticle}
%\documentclass[final,1p,times,twocolumn,authoryear]{elsarticle}
\documentclass[12pt,letterpaper]{article}

\usepackage{parskip} 
\usepackage[margin=3cm]{geometry}
\usepackage{parskip}
%\usepackage{times}
\usepackage{braket}
\usepackage{mathtools} \mathtoolsset{showonlyrefs} %show only referenced equations
%\usepackage[latin1]{inputenc}
\usepackage[utf8]{inputenc}
%\usepackage[T1]{fontenc}
\usepackage{psfrag}
\usepackage[spanish]{babel}
\usepackage{mathrsfs}  % mathscr
\usepackage{bbm} % R Q N Z symbols
\usepackage{url}
\usepackage{natbib}
\usepackage{amsmath,amssymb,amsthm,amsfonts}

% FOR USE WITH THE TODONOTES PACKAGE:
\usepackage{xcolor}
\usepackage[spanish,textwidth=2cm]{todonotes} %todonotes goes after xcolor (messages, what to do!)
\newcommand\todoin[2][]{\todo[inline, caption={2do}, #1]{\begin{minipage}{\textwidth-4pt}#2\end{minipage}}}

\newcommand{\pf}{P_\text{\textnormal{f}}}
\newcommand{\LP}{\underline{P_{\Gamma}}}
\newcommand{\UP}{\overline{P_{\Gamma}}}
\newcommand{\rv}[1]{{\uppercase{#1}}}    %random variable
\newcommand{\ve}[1]{{\boldsymbol{#1}}}
\newcommand{\ma}[1]{{\boldsymbol{#1}}}
\newcommand{\dd}{\operatorname{d} \!}
%\newcommand{\nuevo}[1]{\textsf{\textcolor{red}{#1}}}
\newcommand{\nuevo}[1]{#1}
\def\bigtimes{\mathbin{\vcenter{\hbox{\Large $\times$}}}} %\mbox{\Large$\times$} = \bigtimes
%\newcommand{\nuevo}[1]{{#1}}

\newtheorem{thm}{Theorem}
\newtheorem{cor}{Corollary}

\usepackage[colorlinks=true,citecolor=blue,%
plainpages=false,pdfpagelabels,%
breaklinks]{hyperref}
\hypersetup{pdffitwindow=false}

\usepackage{hypernat} % in this way [sort&compress]{natbib} and hyperref will work together

\usepackage{breakurl}

\title{Deducción de la ecuación $\ma{K}^{(e)} \ve{a}^{(e)} - \ma{f}^{(e)} = \ma{q}^{(e)}$ para el EF de barra isoparamétrico de $n$ nodos}
\date{}
\begin{document}
\maketitle

\section{Se interpola la geometría utilizando las mismas funciones de forma que interpolan el desplazamiento}

Se interpola la geometría utilizando las mismas funciones de forma que se emplearán para interpolar los desplazamientos:
\begin{align}
x^{(e)}(\xi) 
&= N_1(\xi) x_1^{(e)} + N_2(\xi) x_2^{(e)} + \cdots + N_n(\xi) x_n^{(e)} 
= \sum_{i=1}^n N_i(\xi) x_i^{(e)} \\
&= 
\underbrace{\begin{bmatrix}
    N_1(\xi) & N_2(\xi) & \cdots & N_n(\xi)
    \end{bmatrix}}_{\ma{N}(\xi)}
\underbrace{\begin{bmatrix}
    x_1^{(e)} \\ x_2^{(e)} \\ \vdots \\ x_n^{(e)}
    \end{bmatrix}}_{\ma{x}^{(e)}} = \ma{N}(\xi)  \ve{x}^{(e)}
\end{align}

\begin{itemize}
    \item $\ma{N}(\xi)$ se conoce como la \emph{matriz de funciones de forma (locales)}.
    \item $\ve{x}^{(e)}$ es el \emph{vector de coordenadas nodales del elemento} $e$.
\end{itemize}    



Se calcula el jacobiano de la transformación
\begin{align}
J^{(e)}(\xi) &= \frac{\dd x^{(e)}(\xi)}{\dd \xi} = 
\frac{\dd}{\dd \xi}\left(\sum_{i=1}^n N_i(\xi) x_i^{(e)}\right) = \sum_{i=1}^n \frac{\dd N_i(\xi)}{\dd \xi} x_i^{(e)} \\
&= 
\underbrace{\begin{bmatrix}
    \frac{\dd N_1(\xi)}{\dd \xi} & \frac{\dd N_2(\xi)}{\dd \xi} & \cdots & \frac{\dd N_n(\xi)}{\dd \xi}
    \end{bmatrix}}_{\frac{\dd \ma{N}(\xi)}{\dd \xi}}
\underbrace{\begin{bmatrix}
    x_1^{(e)} \\ x_2^{(e)} \\ \vdots \\ x_n^{(e)}
    \end{bmatrix}}_{\ma{x}^{(e)}} \\
&= \frac{\dd \ma{N}(\xi)}{\dd \xi}  \ve{x}^{(e)}
\end{align}
y, utilizando el \emph{teorema de la función inversa}, también la derivada:
\begin{align}
\frac{\dd \xi(x)}{\dd x} = \frac{1}{J^{(e)}(\xi(x))}.
\end{align}


Recuerde que la transformación isoparamétrica es válida siempre y cuando $J^{(e)}(\xi)>0$ en todos los puntos del EF. Esto garantiza que $x(\xi)$ sea una función biyectiva en el intervalo $\xi\in[-1,1]$.


\subsection*{Teorema de la función inversa}
Tomado de \url{https://en.wikipedia.org/wiki/Inverse_function_theorem}.
 
Considere la función $x:\mathbbm{R} \to \mathbbm{R}$, $\xi \mapsto x(\xi)$ la cual 
es continuamente diferenciable (esto es $x \in C^1$), con derivada diferente de cero en el punto $a$ (o sea  $x'(a) = \frac{\dd x(a)}{\dd \xi} \neq 0$). Si $b = x(a)$, el teorema de la función inversa dice que $x$ es invertible en un vecindario de $a$, la inversa $x^{-1}$ es continua y diferenciable cerca a $a$ y 
\[
\bigl(x^{-1}\bigr)'(b) = \frac{1}{x'(a)} = \frac{1}{x'(x^{-1}(b))}
\]
o escrito alternativamente,
\[
\frac{\dd \xi(b)}{\dd x} = \frac{1}{\frac{\dd x(a)}{\dd \xi}} = \frac{1}{\frac{\dd x(\xi(b))}{\dd \xi}}.
\]

\newpage

\section{Se definen los campos vectoriales de desplazamientos y  de desplazamientos virtuales}
Se interpolan los desplazamientos al interior del EF utilizando las mismas funciones de forma que se emplearon para interpolar la geometría:
\begin{align}
u^{(e)}(\xi) 
&= N_1(\xi) u_1^{(e)} + N_2(\xi) u_2^{(e)} + \cdots + N_n(\xi) u_n^{(e)} 
= \sum_{i=1}^n N_i(\xi) u_i^{(e)} \\
&= 
\underbrace{\begin{bmatrix}
N_1(\xi) & N_2(\xi) & \cdots & N_n(\xi)
\end{bmatrix}}_{\ma{N}(\xi)}
\underbrace{\begin{bmatrix}
   u_1^{(e)} \\ u_2^{(e)} \\ \vdots \\ u_n^{(e)}
   \end{bmatrix}}_{\ma{a}^{(e)}} = \ma{N}(\xi)  \ve{a}^{(e)}
\end{align}

Desplazamientos virtuales al interior del elemento:
\begin{align}
\delta u^{(e)}(\xi) 
&= N_1(\xi) \delta u_1^{(e)} + N_2(\xi) \delta u_2^{(e)} + \cdots + N_n(\xi) \delta u_n^{(e)} = \sum_{i=1}^n N_i(\xi) \delta u_i^{(e)} \\
&= 
\underbrace{\begin{bmatrix}
   N_1(\xi) & N_2(\xi) & \cdots & N_n(\xi)
   \end{bmatrix}}_{\ma{N}(\xi)}
\underbrace{\begin{bmatrix}
   \delta u_1^{(e)} \\ \delta u_2^{(e)} \\ \vdots \\ \delta u_n^{(e)} 
   \end{bmatrix}}_{\delta \ma{a}^{(e)}} = \ma{N}(\xi) \delta \ve{a}^{(e)}
\end{align}

$\ma{N}(\xi)$ se conoce como la \emph{matriz de funciones de forma (locales)}.

$\ve{a}^{(e)}$ es el \emph{vector de desplazamientos nodales del elemento} $e$.

$\delta\ve{a}^{(e)}$ es el \emph{vector de desplazamientos virtuales nodales del elemento} $e$.

\newpage

\section{Se definen los campos vectoriales de deformaciones y  de deformaciones virtuales}

Deformaciones al interior del elemento:
\begin{align}
\varepsilon_x^{(e)}(\xi) &= \frac{\dd u^{(e)}(\xi)}{\dd x} =
 \frac{\dd u^{(e)}(\xi)}{\dd \xi}\frac{\dd \xi(x)}{\dd x} = \frac{\dd}{\dd \xi}\left(\sum_{i=1}^n N_i(\xi) u_i^{(e)}\right)\frac{\dd \xi(x)}{\dd x} \\
 &= \left(\sum_{i=1}^n \frac{\dd N_i(\xi)}{\dd \xi} u_i^{(e)}\right) \frac{1}{J^{(e)}(\xi)} = 
\underbrace{\begin{bmatrix}
  \frac{\dd N_1(\xi)}{\dd \xi} & \frac{\dd N_2(\xi)}{\dd \xi} & \cdots & \frac{\dd N_n(\xi)}{\dd \xi}
   \end{bmatrix}}_{\frac{\dd \ma{N}(\xi)}{\dd \xi}} \frac{1}{J^{(e)}(\xi)}
\underbrace{\begin{bmatrix}
   u_1^{(e)} \\ u_2^{(e)} \\ \vdots \\ u_n^{(e)}
   \end{bmatrix}}_{\ma{a}^{(e)}} \\
&= 
\underbrace{\frac{\dd \ma{N}(\xi)}{\dd \xi}%
    \frac{1}{J^{(e)}(\xi)}}_{\ma{B}^{(e)}(\xi)}
\ma{a}^{(e)} = \ma{B}^{(e)}(\xi)  \ve{a}^{(e)}
\end{align}

Deformaciones virtuales al interior del elemento:
\begin{align}
\delta \varepsilon_x^{(e)}(\xi) &= \delta \left(\frac{\dd u^{(e)}(\xi)}{\dd x}\right) = \frac{\dd}{\dd x}\left(\delta u^{(e)}(\xi)\right)
= \frac{\dd}{\dd \xi}\left(\sum_{i=1}^n N_i(\xi) \delta u_i^{(e)}\right)\frac{\dd \xi(x)}{\dd x} \\
&= \left(\sum_{i=1}^n \frac{\dd N_i(\xi)}{\dd \xi} \delta u_i^{(e)}\right)\frac{1}{J^{(e)}(\xi)} = 
\underbrace{\begin{bmatrix}
   \frac{\dd N_1(\xi)}{\dd \xi} & \frac{\dd N_2(\xi)}{\dd \xi} & \cdots & \frac{\dd N_n(\xi)}{\dd \xi}
   \end{bmatrix} \frac{1}{J^{(e)}(\xi)} }_{\ma{B}^{(e)}(\xi)}
\underbrace{\begin{bmatrix}
  \delta u_1^{(e)} \\ \delta u_2^{(e)} \\ \vdots \\ \delta u_n^{(e)} 
   \end{bmatrix}}_{\delta \ma{a}^{(e)}} \\
&= \ma{B}^{(e)}(\xi) \delta \ve{a}^{(e)}
\end{align}

$\ma{B}^{(e)}(\xi)$ se conoce como la \emph{matriz de deformaciones del elemento} $e$.

$\ve{a}^{(e)}$ es el \emph{vector de desplazamientos nodales del elemento} $e$.

$\delta\ve{a}^{(e)}$ es el \emph{vector de desplazamientos virtuales nodales del elemento} $e$.

\newpage
\section{Se define el campo vectorial de esfuerzos}
\begin{align}
\sigma_x^{(e)}(\xi) 
&= E^{(e)}(\xi) \varepsilon_x^{(e)}(\xi) \\
&= E^{(e)}(\xi) \ma{B}^{(e)}(\xi) \ve{a}^{(e)}
\end{align}

Tenga en cuenta que las fuerzas axiales al interior del elemento están dadas por:
\begin{align}
f^{(e)}_{\text{axial}}(\xi) 
&= A^{(e)}(\xi) \sigma_x^{(e)}(\xi) \\
&= \underbrace{E^{(e)}(\xi) A^{(e)}(\xi)}_{\ma{D}^{(e)}(\xi)} \ma{B}^{(e)}(\xi) \ve{a}^{(e)} \\
&= \ma{D}^{(e)}(\xi) \ma{B}^{(e)}(\xi) \ve{a}^{(e)}
\end{align}
$\ma{D}^{(e)}(\xi)$ se conoce como la \emph{matriz constitutiva del elemento} $e$.

$\ma{B}^{(e)}(\xi)$ se conoce como la \emph{matriz de deformaciones del elemento} $e$.

$\ve{a}^{(e)}$ es el \emph{vector de desplazamientos nodales del elemento} $e$.

\newpage
\section{Se reemplaza en el principio de los trabajos virtuales}

A partir del PTV:
\begin{align}
%\iiint_{V^{(e)}} \delta\varepsilon_x^{(e)}(x) \sigma_x^{(e)}(x) \dd V 
\int_0^L \sigma_x(x) A(x) \delta \varepsilon_x(x) \dd x = \int_{x_1^{(e)}}^{x_n^{(e)}} \delta u^{(e)}(x) b^{(e)}(x) \dd x + \sum_{i=1}^n \delta u_i^{(e)} q_i^{(e)}
\end{align}
y siguiendo unos pasos similares a los que se realizaron con el EF de barra de 2 nodos, obtenemos que:
\begin{equation}
\underbrace{\int_{x_1^{(e)}}^{x_n^{(e)}} \ma{B}_{(e)}^T(x) \ma{D}^{(e)}(x) \ma{B}^{(e)}(x) \dd x}_{\ma{K}^{(e)}} \ve{a}^{(e)} 
- \underbrace{\int_{x_1^{(e)}}^{x_n^{(e)}}  \ma{N}_{(e)}^T(x) b^{(e)}(x) \dd x}_{\ma{f}^{(e)}} 
= \ma{q}^{(e)}
\end{equation}
es decir,
\begin{equation}
\ma{K}^{(e)} \ve{a}^{(e)} - \ma{f}^{(e)} = \ma{q}^{(e)}.
\end{equation}

Aquí:
\begin{itemize}
    \item $\ma{K}^{(e)}$ es la \emph{matriz de rigidez} del EF $e$.
    \item $\ve{a}^{(e)}$ es el \emph{vector de desplazamientos nodales} del EF $e$.
    \item $\ma{f}^{(e)}$ es el \emph{vector de fuerzas nodales equivalentes} del EF $e$.    
    \item $\ve{q}^{(e)} = [q_1^{(e)}\ q_2^{(e)}\ \cdots \ q_n^{(e)}]^T$ es el \emph{vector de fuerzas nodales de equilibrio} del EF $e$.
\end{itemize}

En las páginas anteriores se calculó $\ma{N}$, $\ma{B}^{(e)}$ y $\ma{D}^{(e)}$ en función de $\xi$, por lo que es necesario hacer un cambio de variables en las integrales anteriores. De este modo:
\begin{align}
\ma{K}^{(e)} &= \int_{-1}^{+1} \ma{B}_{(e)}^T(\xi) \ma{D}^{(e)}(\xi) \ma{B}^{(e)}(\xi) \frac{\dd x^{(e)}(\xi)}{\dd \xi} \dd \xi\\
\ma{f}^{(e)} &= \int_{-1}^{+1}  \ma{N}_{(e)}^T(\xi) b^{(e)}(\xi) \frac{\dd x^{(e)}(\xi)}{\dd \xi} \dd \xi
\end{align}

Finalmente, se aproximan dichas integrales utilizando una cuadratura de $M$ puntos de Gauss-Legendre:
\begin{align}
\ma{K}^{(e)} &\approx \sum_{j=1}^M \ma{B}_{(e)}^T(\xi_j) \ma{D}^{(e)}(\xi_j) \ma{B}^{(e)}(\xi_j) \frac{\dd x^{(e)}(\xi_j)}{\dd \xi} w_j\\
\ma{f}^{(e)} &\approx \sum_{j=1}^M  \ma{N}_{(e)}^T(\xi_j) b^{(e)}(\xi_j) \frac{\dd x^{(e)}(\xi_j)}{\dd \xi} w_j
\end{align}

NOTA: la matriz $\ma{K}^{(e)}$ es simétrica y semi-positiva definida. Tiene un valor propio nulo asociado con la translación rígida horizontal. El rango de $\ma{K}^{(e)}$ es $n-1$.

Se deja como ejercicio al lector demostrar que:
\begin{align}
\ma{K}^{(e)}_{pq} &\approx \sum_{j=1}^M \frac{\dd N_p(\xi_j)}{\dd \xi} E^{(e)}(\xi_j) A^{(e)}(\xi_j) \frac{\dd N_q(\xi_j)}{\dd \xi} \frac{1}{J^{(e)}(\xi_j)} w_j\\
\ma{f}^{(e)}_p &\approx \sum_{j=1}^M  N_p^{(e)}(\xi_j) b^{(e)}(\xi_j) J^{(e)}(\xi_j) w_j
\end{align}
donde $p, q = 1,\ 2, \ \ldots,\ n$.
\end{document}
