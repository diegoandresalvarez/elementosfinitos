% !TeX spellcheck = en_US
% !TeX document-id = {cf032f19-4427-4a50-b0ba-ddee651fb980}
% !TeX program = txs:///dvi-ps-pdf-chain
% TeX program = txs:///dvi-chain
%DVI -> PS -> PDF
% 10.1016/j.compstruc.2017.04.006 
% OJO VERIFICAR IDIOMA GB

%\documentclass[preprint,12pt,authoryear,letterpaper,times]{elsarticle}
%\documentclass[final,1p,times,twocolumn,authoryear]{elsarticle}
\documentclass[12pt,landscape,letterpaper]{article}

\usepackage{parskip} 
\usepackage[margin=2cm,left=5cm,right=5cm]{geometry}
\usepackage{parskip}
%\usepackage{times}
\usepackage{braket}
\usepackage{mathtools} \mathtoolsset{showonlyrefs} %show only referenced equations
%\usepackage[latin1]{inputenc}
\usepackage[utf8]{inputenc}
%\usepackage[T1]{fontenc}
\usepackage{psfrag}
\usepackage[spanish]{babel}
\usepackage{mathrsfs}  % mathscr
\usepackage{bbm} % R Q N Z symbols
\usepackage{url}
\usepackage{natbib}
\usepackage{amsmath,amssymb,amsthm,amsfonts}

% FOR USE WITH THE TODONOTES PACKAGE:
\usepackage{xcolor}
\usepackage[spanish,textwidth=2cm]{todonotes} %todonotes goes after xcolor (messages, what to do!)
\newcommand\todoin[2][]{\todo[inline, caption={2do}, #1]{\begin{minipage}{\textwidth-4pt}#2\end{minipage}}}

\newcommand{\pf}{P_\text{\textnormal{f}}}
\newcommand{\LP}{\underline{P_{\Gamma}}}
\newcommand{\UP}{\overline{P_{\Gamma}}}
\newcommand{\rv}[1]{{\uppercase{#1}}}    %random variable
\newcommand{\ve}[1]{{\boldsymbol{#1}}}
\newcommand{\ma}[1]{{\boldsymbol{#1}}}
\newcommand{\dd}{\operatorname{d} \!}
%\newcommand{\nuevo}[1]{\textsf{\textcolor{red}{#1}}}
\newcommand{\nuevo}[1]{#1}
\def\bigtimes{\mathbin{\vcenter{\hbox{\Large $\times$}}}} %\mbox{\Large$\times$} = \bigtimes
%\newcommand{\nuevo}[1]{{#1}}

\newtheorem{thm}{Theorem}
\newtheorem{cor}{Corollary}

\usepackage[dvips,%
colorlinks=true,citecolor=blue,%
plainpages=false,pdfpagelabels,%
breaklinks]{hyperref}
\hypersetup{pdffitwindow=false}

\usepackage{hypernat} % in this way [sort&compress]{natbib} and hyperref will work together

\usepackage{breakurl}

\title{Deducción de la ecuación $\ma{K}^{(e)} \ve{a}^{(e)} - \ma{f}^{(e)} = \ma{q}^{(e)}$ empleando funciones de forma locales para el EF de barra de 2 nodos}
\date{}
\begin{document}
\maketitle

\section{Se definen los campos vectoriales de desplazamientos y  de desplazamientos virtuales}

Desplazamientos al interior del elemento:
\begin{align}
u^{(e)}(x) 
&= N_1^{(e)}(x) u_1^{(e)} + N_2^{(e)}(x) u_2^{(e)} 
= \sum_{i=1}^2 N_i^{(e)}(x) u_i^{(e)} \\
&= 
\underbrace{\begin{bmatrix}
N_1^{(e)}(x) & N_2^{(e)}(x)
\end{bmatrix}}_{\ma{N}^{(e)}(x)}
\underbrace{\begin{bmatrix}
   u_1^{(e)} \\ u_2^{(e)}
   \end{bmatrix}}_{\ma{a}^{(e)}} = \ma{N}^{(e)}(x)  \ve{a}^{(e)}
\end{align}

Desplazamientos virtuales al interior del elemento:
\begin{align}
\delta u^{(e)}(x) 
&= N_1^{(e)}(x) \delta u_1^{(e)} + N_2^{(e)}(x) \delta u_2^{(e)} = \sum_{i=1}^2 N_i^{(e)}(x) \delta u_i^{(e)} \\
&= 
\underbrace{\begin{bmatrix}
   N_1^{(e)}(x) & N_2^{(e)}(x)
   \end{bmatrix}}_{\ma{N}^{(e)}(x)}
\underbrace{\begin{bmatrix}
   \delta u_1^{(e)} \\ \delta u_2^{(e)}
   \end{bmatrix}}_{\delta \ma{a}^{(e)}} = \ma{N}^{(e)}(x) \delta \ve{a}^{(e)}
\end{align}

$\ma{N}^{(e)}(x)$ se conoce como la \emph{matriz de funciones de forma del elemento} $e$.

\newpage

\section{Se definen los campos vectoriales de deformaciones y  de deformaciones virtuales}

Deformaciones al interior del elemento:
\begin{align}
\varepsilon^{(e)}(x) = \frac{\dd u^{(e)}(x)}{\dd x} 
&= \sum_{i=1}^2 \frac{\dd N_i^{(e)}(x)}{\dd x} u_i^{(e)} \\
&= 
\underbrace{\begin{bmatrix}
  \frac{\dd N_1^{(e)}(x)}{\dd x} & \frac{\dd N_2^{(e)}(x)}{\dd x}
   \end{bmatrix}}_{\ma{B}^{(e)}(x)}
\underbrace{\begin{bmatrix}
   u_1^{(e)} \\ u_2^{(e)}
   \end{bmatrix}}_{\ma{a}^{(e)}} = \ma{B}^{(e)}(x)  \ve{a}^{(e)}
\end{align}

Deformaciones virtuales al interior del elemento:
\begin{align}
\delta \varepsilon^{(e)}(x) &= \delta \left(\frac{\dd u^{(e)}(x)}{\dd x}\right) = \frac{\dd \delta u^{(e)}(x)}{\dd x} 
= \sum_{i=1}^2 \frac{\dd N_i^{(e)}(x)}{\dd x} \delta u_i^{(e)} \\
&= 
\underbrace{\begin{bmatrix}
   \frac{\dd N_1^{(e)}(x)}{\dd x} & \frac{\dd N_2^{(e)}(x)}{\dd x}
   \end{bmatrix}}_{\ma{B}^{(e)}(x)}
\underbrace{\begin{bmatrix}
  \delta u_1^{(e)} \\ \delta u_2^{(e)}
   \end{bmatrix}}_{\delta \ma{a}^{(e)}} = \ma{B}^{(e)}(x) \delta \ve{a}^{(e)}
\end{align}

$\ma{B}^{(e)}(x)$ se conoce como la \emph{matriz de deformaciones del elemento} $e$.

\newpage
\section{Se define el campo vectorial de esfuerzos}
\begin{align}
\sigma^{(e)}(x) 
&= E^{(e)}(x) \varepsilon^{(e)}(x) \\
&= E^{(e)}(x) \ma{B}^{(e)}(x) \ve{a}^{(e)}
\end{align}

Tenga en cuenta que las fuerzas axiales al interior del elemento están dadas por:
\begin{align}
f^{(e)}_{\text{axial}}(x) 
&= A^{(e)}(x) \sigma^{(e)}(x) \\
&= \underbrace{E^{(e)}(x) A^{(e)}(x)}_{\ma{D}^{(e)}(x)} \ma{B}^{(e)}(x) \ve{a}^{(e)} \\
&= \ma{D}^{(e)}(x) \ma{B}^{(e)}(x) \ve{a}^{(e)}
\end{align}
$\ma{D}^{(e)}(x)$ se conoce como la \emph{matriz constitutiva del elemento} $e$.

\newpage
\section{Se reemplaza en el PTV}
\begin{align}
\iiint_{V^{(e)}} \delta\varepsilon^{(e)}(x) \sigma^{(e)}(x) \dd V = \int_{x_1^{(e)}}^{x_2^{(e)}} \delta u^{(e)}(x) b^{(e)}(x) \dd x + \delta u_1^{(e)} R_1^{(e)} + \delta u_2^{(e)} R_2^{(e)}
\end{align}

Teniendo en cuenta que para un escalar $c$ se tiene que $c = c^T$ tenemos que:
\begin{equation}
\iiint_{V^{(e)}} \underbrace{\delta \ve{a}_{(e)}^T \ma{B}_{(e)}^T(x)}_{\delta\varepsilon_{(e)}^T(x)} \underbrace{E^{(e)}(x) \ma{B}^{(e)}(x) \ve{a}^{(e)}}_{\sigma^{(e)}(x)} \dd V = %\\
%
\int_{x_1^{(e)}}^{x_2^{(e)}} \underbrace{\delta \ve{a}_{(e)}^T \ma{N}_{(e)}^T(x)}_{\delta u_{(e)}^T(x)} b^{(e)}(x) \dd x +
\underbrace{\begin{bmatrix}
   \delta u_1^{(e)} & \delta u_2^{(e)}
   \end{bmatrix}}_{\delta \ma{a}_{(e)}^T}
\underbrace{\begin{bmatrix}
R_1^{(e)} \\ R_2^{(e)}
   \end{bmatrix}}_{\ma{q}^{(e)}}
\end{equation}
Y como los vectores $\ve{a}^{(e)}$ y $\delta \ve{a}^{(e)}$  son independientes de $x$, salen de las integrales:
\begin{equation}
\delta \ve{a}_{(e)}^T \int_{x_1^{(e)}}^{x_2^{(e)}} \iint_{A^{(e)}(x)}  \ma{B}_{(e)}^T(x) E^{(e)}(x) \ma{B}^{(e)}(x) \dd A \dd x\  \ve{a}^{(e)}  %\\
%
- \delta \ve{a}_{(e)}^T \int_{x_1^{(e)}}^{x_2^{(e)}}  \ma{N}_{(e)}^T(x) b^{(e)}(x) \dd x 
- \delta \ma{a}_{(e)}^T \ma{q}^{(e)} = 0
\end{equation}
Se factoriza $\delta \ma{a}_{(e)}^T$ y se tiene en cuenta que el integrando de la primera integral no depende de $A$
\begin{equation}
\delta \ve{a}_{(e)}^T 
\left[ \int_{x_1^{(e)}}^{x_2^{(e)}} \ma{B}_{(e)}^T(x) E^{(e)}(x) A^{(e)}(x) \ma{B}^{(e)}(x) \dd x \ve{a}^{(e)}  
%
- \int_{x_1^{(e)}}^{x_2^{(e)}}  \ma{N}_{(e)}^T(x) b^{(e)}(x) \dd x 
- \ma{q}^{(e)}\right] = 0
\end{equation}
Teniendo en cuenta que $\ma{D}^{(e)}(x) = E^{(e)}(x) A^{(e)}(x)$ y que el vector de desplazamientos virtuales $\delta \ma{a}^{(e)}$ es arbitrario, se tiene que el término entre corchetes debe valer cero, por lo que:
\begin{equation}
\underbrace{\int_{x_1^{(e)}}^{x_2^{(e)}} \ma{B}_{(e)}^T(x) \ma{D}^{(e)}(x) \ma{B}^{(e)}(x) \dd x}_{\ma{K}^{(e)}} \ve{a}^{(e)} 
- \underbrace{\int_{x_1^{(e)}}^{x_2^{(e)}}  \ma{N}_{(e)}^T(x) b^{(e)}(x) \dd x}_{\ma{f}^{(e)}} 
= \ma{q}^{(e)}
\end{equation}
es decir,
\begin{equation}
\ma{K}^{(e)} \ve{a}^{(e)} - \ma{f}^{(e)} = \ma{q}^{(e)}
\end{equation}
\end{document}
